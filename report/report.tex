\documentclass[10pt,a4paper]{report}

\usepackage{amsmath}
\usepackage{amssymb}
\usepackage{mathtools}
\usepackage{color, soul}
\usepackage{array}
\usepackage{listings}
\usepackage{booktabs}
\usepackage{fancyvrb,booktabs}

\newenvironment{tabularverbatim}
 {\VerbatimEnvironment
  \begin{BVerbatim}[baseline=c,formatcom=\setlength{\baselineskip}{\normalbaselineskip}]}
 {\end{BVerbatim}}

\begin{document}
  \section*{WACC Project Report}


  \subsection*{1. The Final Product}
% \textit{An analysis and critical evaluation of the quality of the WACC compiler
% have built. You should consider both whether it meets the functional
% specification of the project and whether you judge that it forms a sound basis
% for future development. You may also wish to address the performance
% characteristics of your compiler.}

  We deemed our implementation of the WACC compiler to have completed the
  functional requirements of the specification to a satisfactory degree, as
  evidenced by the test results, in addition to which we also included
  additional test cases for edge cases that were not covered by the original set.
  Our implementation of the WACC compiler also provided us with a expandable
  platform from with additional functionalities, and language features can be
  added. The lexer and parser can easily be expanded to include new language
  features. The syntactic and semantic analysis, owing to our design choices in
  the implementation of errors as bundles, allows for the easy addition of new
  syntax and semantic error types, under new conditions, whilst maintaining the
  ability to gather all thrown error messages in one compilation attempt.
  Our design choice surrounding machine code generation also allows for
  additional arm instructions to be implemented into the compiler, whether it is
  to replace previous implementations or as part of a new language feature.

  \hl{Performance characteristics of our compiler?}
  We implemented our WACC compiler using Kotlin, which provides a similar
  performance characteristic to Java. We did, however, avoid, where applicable,
  process repetition. For example, we preserve our symbol table after semantic
  analysis and reuses it during code generation.


  \subsection*{2. Project Management}
% \textit{An analysis of the organization of your group and your use of project
% management tools (such as Git). You should describe how your group was
% structured, how you coordinated your work and detail any tools that
% helped/hindered your progress. You should also discuss what went well and what
% you would do differently if you were to do the lab again.}

  \subsubsection*{2.1 Work Distribution}
  The work distribution of the first two milestones in our group is quite
  different from that of the last. After discussing, we noticed that for the
  frontend as well as backend, most of the tasks had to be done sequentially as
  opposed to in parallel. For instance, syntax analysis for the compiler would
  have to be implemented before semantics analysis. Hence, we did pair
  programming as opposed to working separately and merging all the code in the
  end. We would often have productive programming sessions during the week and
  even over weekends. In each session, two people in the group would meet up and
  work on new features together, with one person typing and the other observing,
  and we would constantly switch roles. The advantages this technique brings is
  evident. Firstly, this led to more optimal design choices as the person that
  we worked with could often notice flaws or code that could be improved, which
  would otherwise be neglected if we had programmed alone. Secondly, debugging
  was much more efficient since logical errors could be found more easily when
  two people discussed and communicated their ideas.

  For the last milestone, we decided to implement each new extension features
  separately, due to the fact that most features are standalone and independent
  of others. However, this did lead to an obvious issue, which is that the
  details of implementation of a new feature are only known to the group member
  who worked on it. Despite the drawback, working separately allowed us to
  incorporate more features in a wide range of areas.

  \subsubsection*{2.2 Use of Tools}
  Git was used extensively throughout the project. It primarily served the
  purpose of creating and keeping track of branches for tentative features that
  were yet to be implemented correctly. For instance, new branches were
  initiated for each new extension feature we were planning to incorporate.
  Testing was mainly performed locally as opposed to on Git since we had some
  troubles installing the required packages for testing on our Gitlab runner.

  Gradle is another tool which facilitated our project apart from Git. In the
  beginning, we put a lot of effort into configuring Gradle properly, and it was
  frustrating and quite time-consuming. Once set up correctly, the incremental
  builds provided by Gradle made building and testing our code much quicker.

  \subsubsection*{2.3 Adjustments for the Future}
  There are a few aspects that can be improved regarding work distribution and
  communication in the last milestone. Firstly, the overlap between new language
  features and optimizations for code generation was not dealt with properly,
  which caused corresponding optimizations not to be implemented for some
  extension features, such as traits and \texttt{newtype}. Therefore, our
  planning will need to be more thorough in the future. Secondly, the details of
  implementation in our independent work will have to be communicated to every
  group member so that less confusion will be caused when trying to modify or
  debug someone else’s code.



  \subsection*{3. Design Choices and Implementation Details}
  \textit{An analysis of the design choices that your made during the WACC
  project, including your implementation and language and tool choices (with
  justifications), and any interesting issues you had to overcome during the
  implementation of your compiler. You should discuss the design patterns you
  used when designing your code and why you chose to use them. You might also
  want to provide a system architecture diagram for your compiler to aid this
  discussion.}

  \subsection*{4. Beyond the Specification}
% \textit{An evaluation of your extensions to your WACC compiler. You should
% describe all the language extensions, optimizations or other aspects that you
% have added to your compiler, especially highlighting how these features can be
% accessed or viewed. You should also briefly discuss what future extensions you
% would add to your compiler if you had more time.}

  \subsubsection*{4.1 Various Code Generation Optimizations}
  There are three levels of optimizations which were implemented for the
  extension, which are constant folding, constant propagation and peephole
  optimization. To run the compiler with optimization, simply add the
  \texttt{-o\{n\}} flag as an argument to the compile script, where \texttt{n}
  indicates the desired level of optimization with 0 to 2 corresponding to
  constant folding, constant propagation and peep-hole optimization respectively.
  If optimization is turned on, the program will output the number of lines
  reduced and the optimization rate at the end of compilation.

  We implemented both constant folding and constant propagation on the AST level
  as opposed to the conventional intermediate representation level. We opted for
  this approach since our intermediate representation for backend is not in the
  form of three-address code or SSA and switching from our original
  implementation to a more convenient IR was proven to be too costly. As a
  result, most of the optimizations were done by reconstructing the AST, and
  some of the more sophisticated optimizations such as dead code elimination and
  loop invariant code motion were not implemented.

  \subsubsection*{4.1.1 Constant Folding}
  Constant folding is done by traversing though the AST and evaluating binary
  operations as well as unary operations that contain only constants. Some
  examples are shown as follows

  \begin{center}
    \begin{tabular} {| m{5cm} | m{5cm} |}
      \hline
      Before Constant Folding & After Constant Folding \\
      \hline
      \begin{tabularverbatim}

int x = (1 + 2) * 3

      \end{tabularverbatim}
      &
      \begin{tabularverbatim}
int x = 9
      \end{tabularverbatim}
      \\
      \hline
      \begin{tabularverbatim}

bool b = true && false 

      \end{tabularverbatim}
      &
      \begin{tabularverbatim}
bool b = false 
      \end{tabularverbatim}
      \\
      \hline
      \begin{tabularverbatim}

int x = 10;
if (x < 2 * 6) then
  println 1
else
  println 2
fi

      \end{tabularverbatim}
      &
      \begin{tabularverbatim}
int x = 10;
if (x < 12) then
  println 1
else
  println 2
fi
      \end{tabularverbatim}
      \\
      \hline
    \end{tabular}
  \end{center}


  \subsubsection*{4.1.2 Constant Propagation}
  Similar to constant folding, our implementation of constant propagation
  traverses through the AST and simplifies each node. However, instead of only
  evaluating expressions that consist of only constants, we record the values
  assigned to each variable, substitute its value into an operation if
  applicable and propagate through the whole program. However, substitution will
  not be performed in a situation where the value of a variable cannot be
  determined. In addition, loops and conditional statements are also optimized
  in such a manner so as to guarantee that the post-conditions of these
  statements are defined correctly. Examples are shown as follows 

  \begin{center}
    \begin{tabular}{| m{5cm} | m{5cm} |}
      \hline
      Before Constant Propagation & After Constant Propagation \\
      \hline
      \begin{tabularverbatim}

int x = 1;
int y = x + 3

      \end{tabularverbatim}
       &
      \begin{tabularverbatim}
int x = 1; 
int y = 4
      \end{tabularverbatim}
      \\
      \hline
      \begin{tabularverbatim}
      
int x = 12;
if (x == 12) then
  println "x is 12"
else \newline
  println "x is not 12"
fi

      \end{tabularverbatim}
      &
      \begin{tabularverbatim}
println "x is 12"
      \end{tabularverbatim}
      \\
      \hline
      \begin{tabularverbatim}

int i = 0;
while (i < 10) do
  i = i + 1
done;
println i

      \end{tabularverbatim}
      &
      \begin{tabularverbatim}
int i = 0;
println i
      \end{tabularverbatim}
      \\
      \hline
    \end{tabular}
  \end{center}


  \subsubsection*{4.1.3 Peephole Optimization}
  Unlike the previous techniques, the peephole optimization is done on the
  intermediate representation of the ARM assembly, so that certain patterns can
  be clearly identified and optimized. The algorithm is rather simple as it goes
  through the instructions line by line and try to determine which optimizable
  pattern matches the current instructions. There are four types of patterns
  that are currently simplified, which are demonstrated in the table below

  \begin{center}
    \begin{tabular}{| m{3cm} | m{4cm} | m{4cm} |}
      \hline
      Pattern & Before Optimization & After Optimization \\
      \hline
      Multiply by a number that is power of two &
      \begin{tabularverbatim}

MOV rn, #n
SMULL rm, rn, rm, rn
CMP rn, rm, ASR #31
BLNE p_overflow_error
STR rm, [offset]

      \end{tabularverbatim}
      \newline
      Where n is an integer power of 2
      &
      \texttt{LSLS rm, rm \#$(\log_2 n)$}
      \newline
      \texttt{BLE} \verb|p_overflow_error|
      \newline
      \texttt{STR rm,} \verb|[offset]|
      \\
      \hline
      Divide by a power of 2 &
      \begin{tabularverbatim}

MOV r0, rm
MOV r1, #n
BL p_check_div_by_zero
BL __aeabi_idiv
MOV rx, r0
 
      \end{tabularverbatim}
      \newline
      Where n is an integer power of 2
      &
      \texttt{LSR rm, rm, \#$(\log_2 n)$}
      \\
      \hline
      Load a constant as an argument &
      \begin{tabularverbatim}

MOV rn #n
MOV r0 rn
 
      \end{tabularverbatim}
      &
      \begin{tabularverbatim}
MOV r0 #n

      \end{tabularverbatim}
      \\
      \hline
      Redundant load after store &
      \begin{tabularverbatim}

STR rn, [offset]
LDR rn, [offset]

      \end{tabularverbatim}
      \newline
      Where the two offsets are the same
      &
      \begin{tabularverbatim}
STR rn, [offset]
      \end{tabularverbatim}
      \\
      \hline
    \end{tabular}
  \end{center}


  \subsubsection*{4.2 VS Code Extension}
  Since WACC is a language designed purely for academic purposes for the
  compiler course, we could not find tools to help users code in WACC. Hence,
  our group decided to build an IDE to help programmers code in WACC more
  efficiently. Since it would be too costly to create one from scratch and it
  would not be easy to popularize the tool, we opted for VS Code as a platform 
  to develop and deploy our WACC IDE. To start, we created syntax highlighting
  for WACC. Then, we utilized the sample language server for VS Code extension
  development as a basis for our IDE, and gradually customized the server
  according to the specifications of the WACC language. The core of the IDE is
  our own WACC compiler, as we parsed the output from the compiler and generated
  code diagnostics based on those. In addition, the users could also execute the
  WACC file directly in VS Code. Since it was difficult to append an ARM
  assembly emulator to the extension, we took advantage of the provided
  reference emulator script. Thus, every time a file is to be executed, it will
  first be compiled into assembly by our compiler, then the assembly code is
  sent to the reference compiler. Lastly, the language server parses the output
  generated by the emulator and displays the messages in VS Code. The IDE
  currently has four main features, which are syntax highlighting, syntax and
  semantic diagnostics, auto completion and code execution. It has to be noted
  that the IDE only supports the most basic WACC language with no other features
  added. We decided not to include our own features as we thought it would be
  better for the IDE to be an opensource project which everyone can customize
  and distribute according to their own flavour of WACC. To access the VS Code
  extension, simply search for “WACC language support” in the marketplace and
  install it.


  \subsubsection*{4.3 Ideas for future extension}
  If more time was given, we would possibly redesign the backend to use SSA as
  our IR so that more optimization options such as dead code elimination would
  be easier to implement. Moreover, we would want to make the import feature
  available so that we could build our own standard libraries for WACC.
  
  \hl{(Feel free to add your own thoughts)}

\end{document}
\documentclass[10pt,a4paper,]{report}

\usepackage{amsmath}
\usepackage{amssymb}
\usepackage{mathtools}
\usepackage{color, soul}
\usepackage{array}
\usepackage{listings}
\usepackage{booktabs}
\usepackage{fancyvrb,booktabs}
\usepackage{hyperref}
\usepackage{xcolor}
\usepackage{listings}
\usepackage{pifont}
\usepackage{appendix}
\usepackage{xparse}
\usepackage{geometry}
\geometry{margin=1.4in}

\newenvironment{tabularverbatim}
 {\VerbatimEnvironment
  \begin{BVerbatim}[baseline=c,formatcom=\setlength{\baselineskip}{\normalbaselineskip}]}
 {\end{BVerbatim}}

\NewDocumentCommand{\keyword}{v}{%
\texttt{\textcolor{blue}{#1}}%
}

\NewDocumentCommand{\simplecode}{v}{%
\texttt{\textcolor{black}{#1}}%
}

\begin{document}
  \begin{center}
    \section*{WACC Project Report}
    Group 51 \ding{88} Zhige Yu - zy7218 \ding{86} Siyuan Shen - ss16118 \ding{86} Ruoyu Hu - rh4618
  \end{center}

  \subsection*{1. The Final Product}
% \textit{An analysis and critical evaluation of the quality of the WACC compiler
% have built. You should consider both whether it meets the functional
% specification of the project and whether you judge that it forms a sound basis
% for future development. You may also wish to address the performance
% characteristics of your compiler.}

  We deemed our implementation of the WACC compiler to have completed the
  functional requirements of the specification to a satisfactory degree, as
  evidenced by the test results, in addition to which we also included
  additional test cases for edge cases that were not covered by the original set.
  Our implementation of the WACC compiler also provided us with a expandable
  platform from with additional functionalities, and language features can be
  added. The lexer and parser can easily be expanded to include new language
  features. The syntactic and semantic analysis, owing to our design choices in
  the implementation of errors as bundles, allows for the easy addition of new
  syntax and semantic error types, under new conditions, whilst maintaining the
  ability to gather all thrown error messages in one compilation attempt.
  Our design choice surrounding machine code generation also allows for
  additional arm instructions to be implemented into the compiler, whether it is
  to replace previous implementations or as part of a new language feature.

  We implemented our WACC compiler using Kotlin, which provides a similar
  performance characteristic to Java. We did, however, avoid, where applicable,
  process repetition. For example, we preserve our symbol table after semantic
  analysis and reuses it during code generation.


  \subsection*{2. Project Management}
% \textit{An analysis of the organization of your group and your use of project
% management tools (such as Git). You should describe how your group was
% structured, how you coordinated your work and detail any tools that
% helped/hindered your progress. You should also discuss what went well and what
% you would do differently if you were to do the lab again.}

  \subsubsection*{2.1 Work Distribution}
  The work distribution of the first two milestones in our group is quite
  different from that of the last. After discussing, we noticed that for the
  frontend as well as backend, most of the tasks had to be done sequentially as
  opposed to in parallel. For instance, syntax analysis for the compiler would
  have to be implemented before semantics analysis. Hence, we did pair
  programming as opposed to working separately and merging all the code in the
  end. We would often have productive programming sessions during the week and
  even over weekends. In each session, two people in the group would meet up and
  work on new features together, with one person typing and the other observing,
  and we would constantly switch roles. The advantages this technique brings is
  evident. Firstly, this led to more optimal design choices as the person that
  we worked with could often notice flaws or code that could be improved, which
  would otherwise be neglected if we had programmed alone. Secondly, debugging
  was much more efficient since logical errors could be found more easily when
  two people discussed and communicated their ideas.
  For the last milestone, we decided to implement each new extension features
  separately, due to the fact that most features are standalone and independent
  of others. However, this did lead to an obvious issue, which is that the
  details of implementation of a new feature are only known to the group member
  who worked on it. Despite the drawback, working separately allowed us to
  incorporate more features in a wide range of areas.

  \subsubsection*{2.2 Use of Tools}
  Git was used extensively throughout the project. It primarily served the
  purpose of creating and keeping track of branches for tentative features that
  were yet to be implemented correctly. For instance, new branches were
  initiated for each new extension feature we were planning to incorporate.
  Testing was mainly performed locally as opposed to on Git since we had some
  troubles installing the required packages for testing on our Gitlab runner.

  Gradle is another tool which facilitated our project apart from Git. In the
  beginning, we put a lot of effort into configuring Gradle properly, and it was
  frustrating and quite time-consuming. Once set up correctly, the incremental
  builds provided by Gradle made building and testing our code much quicker.

  \subsubsection*{2.3 Adjustments for the Future}
  There are a few aspects that can be improved regarding work distribution and
  communication in the last milestone. Firstly, the overlap between new language
  features and optimizations for code generation was not dealt with properly,
  which caused corresponding optimizations not to be implemented for some
  extension features, such as \texttt{traits} and \texttt{newtype}. Therefore, our
  planning will need to be more thorough in the future. Secondly, the details of
  implementation in our independent work will have to be communicated to every
  group member so that less confusion will be caused when trying to modify or
  debug someone else’s code.

  \subsection*{3 Design Choices and Implementation Details}
  Before we started our implementation, we did some research over several
  candidate languages, and we chose Kotlin because it is fully interoperable
  with the Java code generated by Antlr, and its more concise syntax, compared
  with Java, can reduce the amount of boilerplate code.

  The general structure of our design is similar to the classical compiler's
  structure: the parser parses the input file and generates an abstract syntax
  tree(AST). The AST is traversed by the semantic analyzer to check for errors
  and generate a symbol table. Then the code generator takes the AST and the
  symbol table, produces a list of intermediate representation of ARM
  instructions. However, all registers in the IR are not real registers, and it
  does not contain any push or pop to normal registers. Thus, before the final
  code emission, a register allocator is responsible for scanning through the
  generated ARM instructions, unify each virtual register to real registers,
  insert push and pop instructions where necessary, and recalculates SP's offset.

  To debug our symbol table, we wrote a \simplecode{dump} method for the symbol
  table that extracts key information from the symbol table, and prints out an
  ascii table. In order to do so, we imported a open-source package named
  \simplecode{asciitable}, and wrote a adapter class \simplecode{TablePrinter}
  to keep all dependencies to the package within the adapter.

  \subsection*{4 Beyond the Specification}

  \subsection*{4.1 Language Extensions}
  We have designed several language extensions to the vanilla WACC language, and
  named our new language AWSL (acronym for "Another WACC-Style Language"). A
  number of backend improvements on code generation are also added to our
  compiler to generate more efficient code.

  \subsubsection*{4.1.1 Basic Syntax and Semantics Improvements}
  AWSL has several improvements in the basic syntax and semantics compared with
  vanilla WACC. Statements in AWSL are separated and optionally ended with
  semicolons. The \keyword{call} keyword for a function call is optional.
  Function calls, array literals and \texttt{newpair} constructors are treated as normal
  expressions. Direct-call statement is also added to AWSL, in case when the
  user wants to call a function and does not care about the return value of it.

  \subsubsection*{4.1.2 Constants}
  Any variables defined with \keyword{const} is said to be constant. Any attempt
  to write to a constant results in a semantic error being thrown at compile time.
  However, write to any member of a \texttt{const struct} or any member of a
  \texttt{const array} is valid. 

  \subsubsection*{4.1.3. Extra Conditional Control Flow}
  AWSL provides support to several extra conditional control flow
  statements/expression, including multi-case if-statements, if-statements
  without else branch, for-loops, when-clauses (see union types) and ternary
  if-expressions. 

  \subsubsection*{4.1.4. User-defined Struct Types, Tagged Union Types and When-clause}
  AWSL allows users to define their own types. User may either defined a simples
  struct type, or a tagged union type.

  A struct type, just like \simplecode{struct} in C, contains one or several
  members. Each member of a struct is mutable and direct-accessible via
  \simplecode{<variable>.<member_name>}.

  A union type, just like ADTs in Haskell, is defined by tags and each tag's
  members. Any direct access to a member of a union is invalid as this operation
  is unsafe (hence each member of a union is implicitly immutable). The only way
  to get value of any member from a union is via a when-clause, which
  pattern-matches a union and gets the values of its members.

  Each definition of a struct type and each tag of any union type implicitly
  generates a constructor function, shares the same name aspect the
  corresponding struct type or union tag.

  \subsubsection*{4.1.5. Generics and Traits}
  Generical types are also allowed in AWSL. A definition of type
  \simplecode{List<A>} allows user to define any variable/expression 
  of type \simplecode{List<int>}, \simplecode{List<bool>},
  \simplecode{List<List<pair(int, char)>>}, etc.

  Trait is an important aspect of AWSL. A type is deemed to be an instance of a
  Trait, provided that all relevant implementations of the trait's required
  functions for that type is in the program, and all of its generics meets their
  requirements declared in the implementation to the trait. Type of any
  functions' parameter can be (or contain) a type variable which implements
  certain traits (or no trait) instead of a concrete type. This provides a
  better abstraction of the code and drastically increases the code reusability. 

  \subsubsection*{4.1.6. Type Inference}
  Instead of direct type-checking, AWSL performs type-inference for expressions
  to validate their types. During semantic analysis, the compiler traverses
  through each sub-expression of the expression, and try to infer a type meets
  the expectation for it. 

  It is crucial that each expression in AWSL must be deduced to either a ground
  type or contains a type variable that is defined in the header of the current
  function. Otherwise, a semantic error will be thrown by the compiler.

  The type-inference feature of AWSL allows users to not specify the type of the
  variable on its declaration. In such case the compiler would attempt to infer
  the type of the variable from the rhs expression.

  \subsubsection*{4.1.7. Full Pair Type}
  In AWSL, nested pair types do not lose any type information. To make AWSL more
  compatible to vanilla WACC, \keyword{pair} in AWSL represents a pair type with
  both of its elements' types being implicit. Notice that the semantic meaning of
  \keyword{pair} in AWSL differs from that in WACC. When \keyword{pair} is
  appeared in the function header, it is impossible to deduce any type
  information about either element of that pair as per the semantics of AWSL.

  \subsection*{4.2. Code Generation Optimizations}
  We implemented both constant folding and constant propagation on the AST level
  as opposed to the conventional intermediate representation level. We opted for
  this approach since our intermediate representation for backend is not in the
  form of three-address code or SSA and switching from our original
  implementation to a more convenient IR was proven to be too costly. As a
  result, most of the optimizations were done by reconstructing the AST, and
  some of the more sophisticated optimizations such as dead code elimination and
  loop invariant code motion were not implemented.

  \subsubsection*{4.2.1	Constant Folding}
  Constant folding is done by traversing through the AST and evaluating binary
  operations as well as unary operations that contain only constants.
  The compiler evaluates and simplifies these expressions to constant values
  before code generation.

  \subsubsection*{4.2.2	Constant Propagation}
  Similar to constant folding, our implementation of constant propagation
  traverses through the AST and simplifies each node. However, instead of 
  only evaluating expressions that consist of only constants, we record the
  values assigned to each variable, substitute its value into an operation if
  applicable and propagate through the whole program. However, substitution will
  not be performed in a situation where the value of a variable cannot be
  determined. In addition, loops and conditional statements are also optimized
  in such a manner so as to guarantee that the post conditions of these
  statements are defined correctly.

  \subsubsection*{4.2.3	Peephole Optimization}
  Unlike the previous techniques, the peephole optimization is done on the
  intermediate representation of the ARM assembly, so that certain patterns can
  be clearly identified and optimized. The algorithm is rather simple as it goes
  through the instructions line by line and try to determine which optimizable
  pattern matches the current instructions.

  \subsection*{4.3 WACC IDE for Visual Studio Code}
  Since WACC is a language designed purely for academic purposes for the
  compiler course, we could not find tools to help users code in WACC. Hence, 
  our group decided to build an IDE to help programmers code in WACC more
  efficiently. Since it would be too costly to create one from scratch and 
  it would not be easy to popularize the tool, we opted for VS Code as a platform
  to develop and deploy our WACC IDE. To start, we created syntax highlighting
  for WACC. Then, we utilized the sample language server for VS Code extension
  development as a basis for our IDE, and gradually customized the server
  according to the specifications of the WACC language. The core of the IDE is
  our own WACC compiler, as we parsed the output from the compiler and generated
  code diagnostics based on those. In addition, the users could also execute the
  WACC file directly in VS Code. Since it was difficult to append an ARM
  assembly emulator to the extension, we took advantage of the provided
  reference emulator script. Thus, every time a file is to be executed, 
  it will first be compiled into assembly by our compiler, then the assembly
  code is sent to the reference compiler. Lastly, the language server parses the 
  output generated by the emulator and displays the messages in VS Code. The IDE
  currently has four main features, which are syntax highlighting, syntax and
  semantic diagnostics, auto completion and code execution. It has to be noted
  that the IDE only supports the most basic WACC language with no other features
  added. We decided not to include our own features as we thought it would be
  better for the IDE to be an opensource project which everyone can customize
  and distribute according to their own flavour of WACC. Link to the extension: 
  \url{https://marketplace.visualstudio.com/items?itemName=YiningShen.wacc-language-support}

  \subsubsection*{4.4 Ideas for future extension}
  Due to the time limit, some of the cool features we came up when designing
  AWSL were axed. For instance, AWSL does not support creating and importing
  modules, and as a consequence, it does not have a standard library. Higher
  order functions was also a language extension we expected to implement for
  AWSL. If more time was given, we would possibly re-design the backend to use
  SSA (e.g. TAC) as our IR so that more optimization options such as dead code
  elimination would be easier to implement.

  \appendixpage

  \subsubsection*{4.2.1 Constant Folding}
  \begin{center}
    \begin{tabular} {| m{5cm} | m{5cm} |}
      \hline
      Before Constant Folding & After Constant Folding \\
      \hline
      \begin{tabularverbatim}

int x = (1 + 2) * 3

      \end{tabularverbatim}
      &
      \begin{tabularverbatim}
int x = 9
      \end{tabularverbatim}
      \\
      \hline
      \begin{tabularverbatim}

bool b = true && false 

      \end{tabularverbatim}
      &
      \begin{tabularverbatim}
bool b = false 
      \end{tabularverbatim}
      \\
      \hline
      \begin{tabularverbatim}

int x = 10;
if (x < 2 * 6) then
  println 1
else
  println 2
fi

      \end{tabularverbatim}
      &
      \begin{tabularverbatim}
int x = 10;
if (x < 12) then
  println 1
else
  println 2
fi
      \end{tabularverbatim}
      \\
      \hline
    \end{tabular}
  \end{center}

  \subsubsection*{4.2.2 Constant Propagation}
  \begin{center}
    \begin{tabular}{| m{5cm} | m{5cm} |}
      \hline
      Before Constant Propagation & After Constant Propagation \\
      \hline
      \begin{tabularverbatim}

int x = 1;
int y = x + 3

      \end{tabularverbatim}
       &
      \begin{tabularverbatim}
int x = 1; 
int y = 4
      \end{tabularverbatim}
      \\
      \hline
      \begin{tabularverbatim}
      
int x = 12;
if (x == 12) then
  println "x is 12"
else
  println "x is not 12"
fi

      \end{tabularverbatim}
      &
      \begin{tabularverbatim}
println "x is 12"
      \end{tabularverbatim}
      \\
      \hline
      \begin{tabularverbatim}

int i = 0;
while (i < 10) do
  i = i + 1
done;
println i

      \end{tabularverbatim}
      &
      \begin{tabularverbatim}
int i = 0;
println i
      \end{tabularverbatim}
      \\
      \hline
    \end{tabular}
  \end{center}


  \subsubsection*{4.2.3 Peephole optimization}
  \begin{center}
    \begin{tabular}{| m{3cm} | m{4cm} | m{4cm} |}
      \hline
      pattern & before optimization & after optimization \\
      \hline
      multiply by a number that is power of two &
      \begin{tabularverbatim}

mov rn, #n
smull rm, rn, rm, rn
cmp rn, rm, asr #31
blne p_overflow_error
str rm, [offset]

      \end{tabularverbatim}
      \newline
      where n is an integer power of 2
      &
      \texttt{lsls rm, rm \#$(\log_2 n)$}
      \newline
      \texttt{ble} \verb|p_overflow_error|
      \newline
      \texttt{str rm,} \verb|[offset]|
      \\
      \hline
      divide by a power of 2 &
      \begin{tabularverbatim}

mov r0, rm
mov r1, #n
bl p_check_div_by_zero
bl __aeabi_idiv
mov rx, r0
 
      \end{tabularverbatim}
      \newline
      where n is an integer power of 2
      &
      \texttt{lsr rm, rm, \#$(\log_2 n)$}
      \\
      \hline
      load a constant as an argument &
      \begin{tabularverbatim}

mov rn #n
mov r0 rn
 
      \end{tabularverbatim}
      &
      \begin{tabularverbatim}
mov r0 #n

      \end{tabularverbatim}
      \\
      \hline
      redundant load after store &
      \begin{tabularverbatim}

str rn, [offset]
ldr rn, [offset]

      \end{tabularverbatim}
      \newline
      where the two offsets are the same
      &
      \begin{tabularverbatim}
str rn, [offset]
      \end{tabularverbatim}
      \\
      \hline
    \end{tabular}
  \end{center}
\end{document}